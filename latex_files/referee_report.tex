\documentclass{article}
\usepackage{amsmath}


\title{Development Economics: Referee Report}
\author{Peter A Tkatchenko}
\date{April 18, 2024}

\begin{document}


\maketitle
The following referee report pertains to the 2011
publication 'Growing Like China,' published in the American
Economic Review by Z. Song, K. Storesletten, and F. Zilibotti.
This referee report proceeds in a series of sections. First,
I will summarize the contents of the paper. Second, I will 
provide comments which highlight the unique contributions
of the paper, as well as criticisms pertaining to its shortcomings.
Lastly, I will outline an empirical and theoretical extension
to the paper which will aim to improve upon the shortcomings
of the publication and address some relevant questions which were left
unanswered by it. Overall, the purpose of this referee report 
is to account for the academic contributions made by the publication,
whilst also identifying remaining knowledge gaps for future research projects.
\\

I will begin with a summary of the publication. The authors begin by 
recounting that the first market-oriented reforms of the Chinese economy
were implemented by Deng Xiaoping in 1978 - the period of structural change beginning 
at this time was underscored by rapid economic growth and resource reallocation within 
the manufacturing sector. Multiple noteworthy stylized facts presented by the authors characterize the 
state of the Chinese economy during this period. The manufacturing sector
undergoes a shift away from the numerical dominance of state-owned firms towards
a meteoric multiplication of domestic private enterprises (DPEs). New domestic private enterprises 
prove to be significantly more productive and profitable than their state-owned counterparts, 
consistently exhibiting 9\% higher profits on average and a TFP advantage of up to a factor of 2.3.
These private enterprises appear to drive the country's accelerated accumulation of foreign reserves, 
as net surpluses become systemicatically larger in DPE-dense provinces. The flourishing 
of private enterprises drives up entrepreneurial incomes, which quickly outpace stagnating worker wages and 
precipitate a sudden widening of the income distribution. The relative success of the private sector 
over the course of the transition stands in contrast to the financial and contractual imperfections 
which systematically repress DPE development. The authors emphasize that largely state-owned Chinese banks provide preferential 
credit market access to state-owned enterprises, which isn't otherwise available to their 
private counterparts. As a result, state-owned enterprises finance a significantly larger share of their investments
through bank loans as compared to DPEs, the latter having to rely largely on retained earnings. Despite being less efficient,
preferential treatment allows state-owned enterprises to dominate capital-intensive production, leaving DPEs to specialize
in the labor-intensive sector. Overall, the incommensurability of constant investment-driven growth, non-decreasing returns to capital,
and a growing foreign surplus during the Chinese transition has been labelled the "allocation puzzle." 
\\

The author's propose a baseline model of the Chinese economic transition in order to explain the occurrence of 
the "allocation puzzle." The authors posit two firm types - financially integrated (F) and entrepreneurial (E) - both
hiring workers and one manager. Financially integrated firms are owned by intermediaries, whereas entrepreneurial
firms are owned by entrepeneurs that possess superior skills and can operate more productive technologies. Due to 
a management agency problem, F firm owners cannot effectively monitor their managers and must centralize their 
corporate governance, whereas entrepreneurs can effectively monitor their managers and delegate authority to them. 
As a result, E firm production is augmented by a factor $\chi$. Banks collect workers' savings and invest them in 
domestic loans and foreign bonds. Despite E firms' productive edge, informational asymmetries between entrepreneurs
and banks leave E firms credit constrained, unable to borrow to the same degree as F firms. Within this theoretical 
context, the Chinese economic transition proceeds as follows. Assuming a sufficiently large productivity gap
betwen E firms and F firms, the employment, capital and output of E firms grow at a constant rate throughout 
the transition. The equilibrium dynamics of F firms are characterized residually in relation to those of 
the E firms. As E firms' employment share increases, the employment share of F firms gradually decreases. As E firms
expand, F firms' growth rate of capital first increases before decreasing and becoming negative. At the end of the transition,
all F firms exit the economy, leaving only E firms to employ and produce output. The dynamics of this model capture well
the empirical reality of the Chinese transition and address all three aspects of the "allocation puzzle." In the theory,
GDP per worker grows at an accelerating rate due to the aggregate TFP gains that accrue as capital and labor gradually 
reallocate towards more productive E firms. The average rate of return to capital increases throughout the transition due to a composition effect,
since the rates of return to capital in E and F firms are constant and the more productive E firms slowly overtake the F firms 
as time progresses. At the same time, the aggregate demand for bank loans decreases as F firms shrink, meaning 
that banks lose their preferred clients and are forced to reorient their portfolios towards foreign bonds. This 
results in China's growing foreign surplus.  
\\

Although the authors' baseline model accounts well for the empirical reality of the Chinese economic transition 
and seemingly solves the "allocation puzzle," a number of questionable assumptions and empirical contradictions remain.
The authors highlight that the present model assumes a competitive and frictionless labor market, which facilitates
the transition of workers from F firms to E firms. In reality, the Chinese labor market significantly deviates from the competitive 
and frictionless abstraction - in particular, the hukou household registration system significantly inhibits the mobility
of individual workers and slows the rate of structural change. Although this arguably significant friction is absent from the 
present paper's model, it is fair to say that the primary objective of the publication was to reconcile certain seemingly incommensurable 
stylized facts with two firm-centered phenomena, namely preferential credit access for state-owned enterprises and the 
productive superiority of private firms. As such, adding an imperfect labour market would be a relatively simple extension which falls 
outside the relevant scope of the present paper. More concerning is the fact that the baseline model predicts a falling rate of investment,
whereas empirically no such falling investment rate is observed during the Chinese transition. The authors posit that this falling 
investment rate can be corrected in the model by inducing generalized capital deepening through a reduction in financial frictions during the transition.
However, the introduction of this mechanism generates falling averate rates of return for both F and E firms in the calibrated model, although 
the opposite is observed in reality. The authors posit the presence of additional sources of efficiency gains 
within F and E firms which offset the decreasing returns. Unfortunately, the theoretical extension that 
they suggest to account for this is a relatively ad hoc addition of within-group heterogeneity, which can't be applied to 
F firms without also relaxing the assumption about their preferential treatment within the credit market. In short, 
the authors posit a brief theoretical extension to account for an empirical inconsistency, only to generate another empirical 
inconsistency, to which they posit a second extension that would detract from one of the central theoretical elements 
of the present publication. 
\\

I will now suggest an extension to the baseline model of the publication that should correct the fall in the rate 
of investment and average rates of return, whilst also relaxing a core assumption for a more thorough analysis.
Rather than positing an exogenous rate of technical change, the extended model assumes that firm owners have the option 
of investing in technologically innovative research in each period. The research process takes two forms: basic 
science research and applied science research. Both research processes produce technological innovations which 
augment the technology factors of firms' production functions. However, technological innovations produced from basic 
science research are more fundamental and have a greater augmentative impact compared to that of technological 
innovations produced from applied science research. Furthermore, basic science research is a significantly more 
capital intensive process as compared to applied science research. Firms are monopolistically competitive and 
generate rents with which they fund basic and applied scientific research. The model preserves the present publication's assumptions 
about capital market imperfections and the productive superiority of E firms. Similarly to the present publication's two-sector 
extension, F firms will specialize in the production of capital-intensive basic science research outputs, whereas E firms will specialize 
in the production of labor-intensive basic science research outputs. The constant augmentation of F and E firms' technology factors 
through investment in scientific research generates the additional efficiency gains needed to offset decreasing returns and 
match the relevant empirical moments. 


























\end{document}