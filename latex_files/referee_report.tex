\documentclass{article}
\usepackage{amsmath}


\title{Development Economics: Referee Report}
\author{Peter A Tkatchenko}
\date{April 18, 2024}

\begin{document}


\maketitle
The following referee report pertains to the 2011
publication 'Growing Like China,' published in the American
Economic Review by Z. Song, K Storesletten, and F. Zilibotti.
This referee report proceeds in a series of sections. First,
I will summarize the contents of the paper. Second, I will 
provide comments which highlight the unique contributions
of the paper, as well as criticisms pertaining to its shortcomings.
Lastly, I will outline an empirical and theoretical extension
to the paper which will aim to improve upon the shortcomings
of the publication and address some relevant questions which were left
unanswered by it. Overall, the purpose of this referee report 
is to account for the academic contributions made by the publication,
whilst also identifying remaining knowledge gaps for future research projects.
\\

I will begin with a summary of the publication. The authors begin by 
recounting that the first market-oriented reforms of the Chinese economy
were implemented by Deng Xiaoping in 1978 - the period of structural change beginning 
at this time was underscored by rapid economic growth and resource reallocation within 
the manufacturing sector. Multiple noteworthy stylized facts presented by the authors characterize the 
state of the Chinese economy during this period. The manufacturing sector
undergoes a shift away from the numerical dominance of state-owned firms towards
a meteoric multiplication of domestic private enterprises (DPEs). New domestic private enterprises 
prove to be significantly more productive and profitable than their state-owned counterparts, 
consistently exhibiting 9\% higher profits on average and a TFP advantage of up to a factor of 2.3.
These private enterprises appear to drive the country's accelerated accumulation of foreign reserves, 
as net surpluses become systemicatically larger in DPE-dense provinces. The flourishing 
of private enterprises drives up entrepreneurial incomes, which quickly outpace stagnating worker wages and 
precipitate a sudden widening of the income distribution. The relative success of the private sector 
over the course of the transition stands in contrast to the financial and contractual imperfections 
which systematically repress DPE development. The authors emphasize that largely state-owned Chinese banks provide preferential 
credit market access to state-owned enterprises, which isn't otherwise available to their 
private counterparts. As a result, state-owned enterprises finance a significantly larger share of their investments
through bank loans as compared to DPEs, the latter having to rely largely on retained earnings. Despite being less efficient,
preferential treatment allows state-owned enterprises to dominate capital-intensive production, leaving DPEs to specialize
in the labor-intensive sector. Overall, the incommensurability of high investment-driven growth, non-decreasing returns to capital,
and a growing foreign surplus during the Chinese transition has been labelled the "allocation puzzle." 










\end{document}